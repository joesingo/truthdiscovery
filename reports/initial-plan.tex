\documentclass{article}
\author{
    Joe Singleton (student)\\
    \and
    Richard Booth (supervisor)
}
\date{January 2019}
\title{
    Initial Plan: Implementation and Analysis of Truth-Discovery Algorithms\\
    \large CM3203 One Semester Individual Project -- 40 credits
}

% Packages:
\usepackage{xcolor}
\usepackage{multicol}
\usepackage{cite}
\usepackage{hyperref}

% Settings
\hypersetup{
	colorlinks,
	citecolor={blue!80!black},
	urlcolor={blue!80!black},
	linkcolor={blue!80!black}
}

\begin{document}
\maketitle
% \begin{multicols}{1}

\section{Project Description}

Truth-discovery algorithms are algorithms that aim to return the true state of
the world (the `facts') given an input consisting of various, possibly
conflicting, reports from different sources of unknown trustworthiness and
reliability. A main characteristic of these algorithms is that they are able to
infer -- from the input alone -- not only belief values associated to different
statements (reflecting the likelihood that they correspond to the truth), but
also some measure of trust over the various sources. Moreover each of these two
different kinds of values should cohere with each other, so that a statement
receives a high belief value if it is backed up by highly trustworthy sources,
while a source receives a high trust value if it provides highly believable
statements.

The design of such algorithms has received an increasing amount of attention in
recent years, especially with regard to aggregation of information on the web.
However the emphasis has been on practical aspects (speed, efficiency
etc\ldots) rather than theoretical foundations. Furthermore the whole process
is usually regarded as a `one-shot' affair, in which it is assumed that all
relevant information has been provided upfront.

This project will have both practical and theoretical components. On the
practical side, a selection of truth-discovery algorithms from the literature
will be implemented.

The implementation will provide a uniform interface for users to run various
truth-discovery algorithms on their own data and evaluate the performance of
each algorithm, both in terms of efficiency and how well the true `facts' are
discovered (in cases where truth or reliability values are already known).

For the theoretical component, some basic axioms of truth-discovery algorithms
will be identified, and the implemented algorithms will be compared against
each other with regard to whether these properties are satisfied.

For example, a basic principle of truth-discovery algorithms is that a source
with highly believable statements receives a high trust value and vice versa;
in this project we will attempt to express this principle in a more precise way
as an axiom of truth-discovery.

Other axioms may concern the changes in trust/belief scores resulting from
small changes the source/claim network.

It may additionally be possible find a sound and complete axiomatisation of a
particular algorithm; i.e. an algorithm $X$ could be characterised by a list of
axioms such that any algorithm $Y$ satisfies these axioms if and only if $X=Y$.
This has been done for the \emph{PageRank} algorithm for the ranking of web
pages in \cite{altman}.

\section{Project Aims and Objectives}
\label{sec:aims}

In this project I aim to:

\begin{itemize}

\item Implement between three and six truth-discovery algorithms in Python.
Command line, Python API and web-based interfaces (time permitting) will be
provided. The user will be able to provide their own input data and select
their desired output format. Well known algorithms that could be implemented
include \emph{Hubs and Authorities} \cite{kleinberg} (adapted to
truth-discovery in \cite{pasternack}), \emph{Average$\cdot$Log},
\emph{Investment}, \emph{PooledInvestment}{\cite{pasternack}},
\emph{TruthFinder}{\cite{yin_han_yu}}, \emph{Cosine}, \emph{2-Estimates} and
\emph{3-Estimates}{\cite{galland}}.

\item Allow users to provide data incrementally, and look at the fluctuation of
the output belief and trust values when sources provide their information
gradually over time instead of all in one go.

\item Apply the implemented algorithms to synthetic and real-world datasets,
and compare the performance of each against a baseline method (e.g. majority
voting).

\item Analyse the properties of existing algorithms by identifying some basic
axioms of truth-discovery. Each of the implemented algorithms will then be
compared with regard to whether they satisfy these axioms. For example, it
would be interesting to identify an axiom for each algorithm that distinguishes
it from the other algorithms. Additionally, the plausibility and intuition
behind each axiom will be discussed.

\item Focus specifically on the \emph{Hubs and Authorities} \cite{kleinberg,
pasternack}, algorithm, and compile a list of sound properties that is complete
as possible, as time permits.

\end{itemize}

\section{Work Plan}

The main deliverables of this project are:
\begin{itemize}

\item A software implementation of a selection of truth-discovery algorithms,
with documentation.
\item Real-world and synthetic datasets applicable to truth-discovery, and a
comparison of the performance of the implemented algorithms on each dataset.
\item A list of five to ten axioms for truth-discovery, with discussion of
their interpretation and plausibility.
\item A final report with detailed discussion of the software implementation,
the algorithms implemented, and the identified axioms.

\end{itemize}

Below is a list of milestones for progress towards delivering the above and
achieving the aims of section \ref{sec:aims}, and dates by which I aim to
complete these milestones.

Note that although there are dependencies between some of the deliverables
listed above (e.g. the report cannot be written before the software
implementation), the plan below assumes that work on certain independent tasks
will overlap.\\

\textbf{By 8th February (Friday week 2)}:
\begin{itemize}
\item A literature review has been carried out to identify which algorithms
from the literature should be implemented. Enough background knowledge has been
obtained to start working on the implementation.
\item The forms of input/output data for the implementation has been decided on.
\end{itemize}

\textbf{By 15th February (Friday week 3)}:
\begin{itemize}
\item Work on the Python implementation has started.
\end{itemize}

\textbf{By 22nd February (Friday Week 4)}:
\begin{itemize}
\item First review meeting with supervisor held.
\item Progress towards axioms: a common framework for truth-discovery has been
set out, so that the developed axioms will be applicable to all algorithms.
\end{itemize}

\textbf{By 8st March (Friday week 6)}:
\begin{itemize}
\item Base python implementation finished for all algorithms. It should be
possible for a user to give their own input data, and receive output using a
simple command-line interface or by using an API from Python code.
\item One or two basic axioms for truth-discovery have been identified.
\end{itemize}

\textbf{By 15st March (Friday week 7)}:
\begin{itemize}
\item Implementation supports viewing the fluctuation of output belief when
providing data incrementally (this could be in the form of graphs or tables).
\end{itemize}

\textbf{By 22nd March (Friday week 8)}:
\begin{itemize}
\item Work on a web interface for the Python implementation has started.
\item An interesting real-world dataset has been identified, and a synthetic
dataset produced. The implemented algorithms have been applied to each, and
conclusions drawn regarding the performance of each algorithm in both
real-world and synthetic cases.
\item Second review meeting with supervisor held.
\end{itemize}

\textbf{By 5th April (Friday week 10)}:
\begin{itemize}
\item Progress made on initial sections of final report, ahead of the Easter
break.
\item The web interface is complete and provides the same functionality as the
command-line interface.
\item Documentation for the implementation, including API usage, has been created.
\end{itemize}

\textbf{By 12th April (Friday week 11)}:
\begin{itemize}
\item Five to ten axioms have been identified, and each algorithm compared with
respect to whether they satisfy the axioms.
\end{itemize}

\textbf{By 10th May (Friday week 12)}:
\begin{itemize}
\item Final report written and submitted.
\end{itemize}

% \end{multicols}

\bibliography{references}{}
\bibliographystyle{plain}

\end{document}
