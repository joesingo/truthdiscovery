\documentclass[../main.tex]{subfiles}
\begin{document}

There is an increasing amount of data available in today's world, particularly
from the vast number of pages on the web, user-submitted content on social
media platforms and blogs, and sensor data from scientific instruments and the
ever-growing Internet of Things. Data can be found in many different formats,
from structured datasets to natural language articles, and from many disparate
sources, e.g. news outlets, businesses, scientific institutions, and members of
the public.

With this abundance of data, it is common to find information about a single
object from multiple sources. An inherent problem faced when using such data is
that different sources may provide \emph{conflicting information} for the same
object.

Conflicts have a variety of causes, including out of date information, poor
quality data sources, errors in information extraction (when parsing natural
language text, for example), and deliberate spread of misinformation. When it
comes to finding information about an object with conflicting reports, the
question is this: \emph{which} information should we accept as correct?

Truth discovery has emerged as a topic aiming to tackle this problem of
determining what to believe by considering also the \emph{trustworthiness} of
sources. A main principle in many approaches to truth discovery is that
believable claims are those that are made by trustworthy sources, and that
trustworthy sources generally make believable claims.

This project has both practical and theoretical components. On the practical
side, we develop a robust software framework in Python for truth discovery,
which supports running truth discovery algorithms on real-world datasets and
evaluating their performance. A few popular algorithms from the literature are
implemented, although the framework aims to be easily extendible,
well-documented and well-tested so as to allow more algorithms and features to
be implemented in the future. The framework will provide a uniform interface
for users and researchers in truth discovery to run different algorithms on
datasets, and compare behaviour between algorithms. Additionally, it could be
used to aid development of new algorithms by evaluating performance against a
number of existing algorithms. To demonstrate its capabilities, some basic
analysis and evaluation of the implemented algorithms is performed.

For the theoretical component, we define a formal mathematical framework for
truth discovery, highlighting parallels with other areas in the literature,
especially the theory of voting in social choice \cite{handbook_voting}.
Following the axiomatic approach used in social choice, we look for axioms
(desirable properties) of truth discovery algorithms expressed in the formal
framework, and consider which axioms are satisfied by a particular algorithm,
namely \emph{Sums} \cite{pasternack}. As well as providing some immediate
results, this provides foundations for further theoretical work on truth
discovery.

\todo{Explain organisation of the report}

\end{document}
